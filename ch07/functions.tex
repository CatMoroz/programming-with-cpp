% !TEX encoding   = UTF8
% !TEX spellcheck = ru_RU
% !TEX root = ../seminars.tex

%%============================================
\chapter{Технические детали: функции и прочее}
%%============================================

%%=====================================
\section{Система сборки \texttt{Cmake}}
%%=====================================
В~прошлый раз для~сборки исполняемого файла программы <<калькулятор>> мы вызывали следующую команду:

\console`$ g++ -o bin/calc -Ilib -pedantic -Wall -Wextra 06/calculator.cpp`

Для~проектов с~более сложной структурой, состоящих из~множества файлов, подключающих внешние библиотеки, имеющих разные варианты конфигурации сборки, намного удобнее воспользоваться системой сборки, например, простым \name{Make}-ом или более мощным и переносимым между платформами \name{CMake}-ом.

По~сути, для~системы сборки мы тоже некоторым образом должны указать детали всей цепочки сборки проекта от~компилятора и его флагов до~исходных файлов с~кодом.



%%===================================
\paragraph{Установка \texttt{CMake}.}
%%===================================
Скачайте \href{\cmakeurl}{архив}\footnote{Дистрибутив CMake: \nolinkurl{\cmakeurl}} \code{CMake} и распакуйте его (например, рядом с~каталогом, куда был распакован \code{MinGW}). Добавьте путь к~папке \code{bin} в~системные пути поиска (переменная \code{PATH}, см.~страницу~\pageref{sect:winSetup}).

Для~поддержки \name{CMake}-а в~\name{VS\,Code} необходимо доставить плагин \textenglish{VS Code Tools} от~Microsoft.

Поскольку данный плагин рассматривает рабочее пространство (\textenglish{workspace}) как один проект, то необходим общий проект, который будет включать в~себя дочерние проекты. Поэтому нам придётся создать проект с~нетривиальной структурой. К~счастью, это не~сильно усложнит нашу задачу.

Корневой и дочерние проекты \name{CMake} называются одинаково \code{CMakeLists.txt}, отличаясь лишь различным положением в~файловой системе.



%%======================
\paragraph{Общий проект}
%%======================
необходимо разместить в~корне рабочего пространства (у~нас это каталог \code{projects}). В~этот проект мы добавим общие настройки, такие как версия стандарта языка~\lang{C++} и дополнительные флаги компилятора (включим всевозможные предупреждения).

\cmakefile[firstline=3, lastline=5]{projects/CMakeLists.txt}

Далее, добавим в~стандартные пути каталог с~нашими библиотеками.

\cmakefile[firstline=7, lastline=11]{projects/CMakeLists.txt}

И, наконец, добавим наш первый дочерний проект, указав путь к~нему. (Создайте указанный каталог для~проекта.)

\cmakefile[firstline=13, lastline=13]{projects/CMakeLists.txt}



%%================================
\paragraph{Проект <<калькулятор>>}
%%================================
размещаем в~каталоге \code{projects/07/calculator}. Ещё раз повторим, что проектный файл создаётся один для~каждого каталога и поэтому на~всех уровнях вложения называется одинаково \code{CMakeLists.txt}. Эта идея наследуется от~\name{Make}.

Создадим локальную переменную \code{TARGET}. Она будет использоваться несколько раз. Будет удобно копировать проект для~других задач и заменять имя проекта только один раз в~этом месте.

\cmakefile[firstline=3, lastline=3]{projects/07/calculator/CMakeLists.txt}

Укажем имя проекта и язык программы. Для получения значения локальной переменной используется синтаксис \code{\$\{TARGET\}}.

\cmakefile[firstline=5, lastline=5]{projects/07/calculator/CMakeLists.txt}

Сообщим системе сборки, что нам нужен исполняемый файл и что исходный код расположен в~одном файле. (Если файлов несколько, просто перечисляем их через пробел или с~новой строки.)

\cmakefile[firstline=7, lastline=10]{projects/07/calculator/CMakeLists.txt}

В~завершение попросим систему сгенерировать инструкции для~установки целевого файла в~заданный каталог. Для~упрощения понимания, можно считать, что файл будет просто скопирован\footnote{В~реальности могут выполняться дополнительные действия в~зависимости от~целевого файла и целевой платформы.}.

\cmakefile[firstline=12, lastline=12]{projects/07/calculator/CMakeLists.txt}



%%==========================================
\paragraph{Как назначить каталог установки?}
%%==========================================
В~настройки рабочего пространства (\textenglish{Workspace Settings JSON}) необходимо добавить строку:

\jsonfile[firstline=5, lastline=5]{projects/.vscode/settings.json}

Чтобы выполнить установку, необходимо явно вызвать команду: \code{Ctrl+Shift+P} и в~строке поиска набрать \code{cmake\,install}.

Таким образом, исполняемый файл окажется в~каталоге \code{projects/bin}, как и ранее, когда мы непосредственно вызывали компилятор из~консоли.



%%================
\paragraph{P.\,S.}
%%================
Поиск компилятора, настройку флагов компиляции и массу другой работы \code{CMake} выполнит вместо нас в~процессе конфигурации. Этим процессом можно управлять. Однако, в~данный момент, нас вполне устроят базовые пресеты (\textenglish{Release, Debug}). Подробнее про систему сборки \code{CMake} можно почитать в~\href{\cmakedocurl}{документации}\footnote{Документация CMake: \nolinkurl{\cmakedocurl}}.



%%====================================
\section{Распределение кода по файлам}
%%====================================
Разместите логически отдельные части калькулятора в~разных файлах, опираясь на~материал \textbookref{главы~8}.

\begin{flushleft}\begin{tabular}{ll}
    \toprule
    \code{token.h}   & Лексемы и поток ввода лексем, константы \\
    \code{token.cpp} & \\[0.5em]

    \code{variable.h}   & Переменные и таблица переменных (символов) \\
    \code{variable.cpp} & \\[0.5em]

    \code{calculator.cpp} & Цикл вычислений, реализация грамматики \\
    \bottomrule
\end{tabular}\end{flushleft}

Помните, что заголовочные файлы должны содержать лишь объявления и константы, чтобы при~включении в~другие единицы трансляции не~происходило переопределение функций и глобальных переменных.

Как правило, содержимое заголовочного файла обрамляется директивами:

\begin{cppcode*}{linenos=false}
#ifndef TOKEN_H
#define TOKEN_H 1

// all code must be here

#endif // #ifndef TOKEN_H
\end{cppcode*}

\noindent Это необходимо во~избежание повторного включения кода из~файла \code{token.h} в~ту же единицу трансляции. (Когда и к~чему это может привести?) Такая ситуация неизбежно возникает в~любой достаточно большой программе.

Директива \cppinline/#ifndef/ проверяет определено ли имя \cppinline/TOKEN_H/. Если да, то она игнорирует весь код вплоть до~завершающей парной директивы \cppinline/#endif/. Если нет, тогда код включается, а \cppinline/TOKEN_H/ определяется при~помощи \cppinline/#define/, чтобы при~повторном подключении файла этот код пропускался. Имя макроса обычно формируют по~названию исходного заголовочного файла и набирают в~верхнем регистре.

Приведённый выше способ является наиболее переносимым между различными платформами. Некоторые компиляторы дополнительно поддерживают директиву, которая, по~сути, служит той же цели:

\cpp/#pragma once/



%%===========================
\section{Замена потока ввода}
%%===========================
Сделайте поток лексем явным параметром функций, как требуется в~упражнении~1 из~\textbookref{главы~8}. Например, функция \code{statement()} будет выглядеть так:

\begin{cppcode*}{linenos=false}
double statement (Token_stream& ts)
{
  Token t = ts.get();
  switch (t.kind)
  {
    case let:
      return declaration(ts);
    default:
      ts.putback(t);
      return expression(ts);
  }
}
\end{cppcode*}

\noindent Глобальную переменную \code{ts} можно (и нужно) сделать локальной:
\begin{cppcode*}{linenos=false}
void calculate ()
{
  Token_stream ts;

  while (true)
  try
  {
// as before

    out << result << statement(ts) << endl;
  }
  catch (runtime_error& e)
  {
// ...
    clean_up_mess(ts);
  }
}
\end{cppcode*}

\noindent Добавьте в~класс \code{Token\_stream} ссылку на~поток ввода и соответствующий конструктор для~инициализации. По~умолчанию, как и ранее, используйте стандартный поток ввода \code{cin}:

\begin{cppcode*}{linenos=false}
class Token_stream
{
  public:
  Token_stream() : in{ cin } {/* empty body */}
  Token_stream(istream& s) : in{ s } {/* empty body */}

// as before

  private:
// ...
  istream& in;
};
\end{cppcode*}

\noindent Определения методов класса следует поправить, заменив \code{cin} на~\code{in} иное подходящее имя.



%%================
\paragraph{P.\,S.}
%%================
После всех изменений, необходимо вновь прогнать тесты (см.~раздел на~странице~\pageref{sect:autotests}) и убедиться, что калькулятор остался в~рабочем состоянии.



%%==================================
\section{\texttt{constexpr} функции}
%%==================================
Немного дополним материл \textbookref{главы~8} по~данной теме\footcite[раздел 3.9]{Meyers:2016:ru}.

\begin{itemize}[itemindent=*, leftmargin=0pt]
    \item Функции, объявленные как \cppinline/constexpr/, могут использоваться в~контекстах, требующих константы времени компиляции. Если значения передаваемых вами аргументов в~\code{соnstехрr}-функцию в~таком контексте известны во~время компиляции, результат функции будет вычислен в~процессе компиляции. Если любое из~значений аргументов неизвестно во~время компиляции, ваш код будет отвергнут.

    \item Когда \code{соnstехрr}-функция вызывается с~одним или несколькими значениями, неизвестными во~время компиляции, она действует так же, как и обычная функция, выполняя вычисления во~время выполнения. Это означает, что вам не~нужны две функции для~выполнения одних и тех же операций, одной "--- для~констант времени компиляции, другой "--- для~всех прочих значений. Функция, объявленная как
    \code{constexpr}, выполняет их все.
\end{itemize}

Поскольку функции \code{constexpr} должны быть способны возвращать результаты во~время компиляции при~вызове со~значениями времени компиляции, на~их реализации накладываются ограничения. Эти ограничения различны в~\lang{C++11} и~\lang{C++14}.

В~\lang{C++11} функции \code{constexpr} могут содержать не~более одной выполнимой инструкции "--- \cppinline/return/. Это выглядит более ограничивающим, чем является на~самом деле, поскольку для~повышения выразительности \code{соnstехрr}-функций можно использовать две хитрости. Во-первых, можно применять условный оператор \code{"? :"} вместо инструкции \cppinline/if-else/, а во-вторых, вместо циклов можно использовать рекурсию. Таким образом, функция \code{pow} может быть реализована следующим образом:

\begin{cppcode*}{linenos=false}
constexpr int pow (int base, int exp) noexcept
{
  return exp == 0 ? 1 : base * pow(base, exp - 1);
}
\end{cppcode*}

Этот код работает, но только очень непритязательный функциональный программист сможет назвать его красивым. В~\lang{C++14} ограничения на~\code{constexpr}-функции существенно слабее, так что становится возможной следующая реализация:

\begin{cppcode*}{linenos=false}
constexpr int pow (int base, int exp) noexcept  // C++14
{
  auto res = 1;
  for (int i = 0; i < exp; ++i)
    res *= base;
  return res;
}
\end{cppcode*}



%%=====================================
\section{Сортировка пар (имя, возраст)}
%%=====================================
Приведём пример реализации программы из~упражнения~7 \textbookref{главы~8}. Будем придерживаться контекста исходного задания из~книги и использовать лишь изученные нами средства.

В~начале мы должны отсортировать вектор имён, а затем синхронизировать его с~вектором возрастов, используя копии исходных массивов.

\cppfile[firstline=48, lastline=56]{projects/07/name_age.cpp}

Чтобы синхронизировать пары, будем брать имена из~нового, сортированного, массива и искать их позицию в~старом массиве-копии. Взяв элемент из~старого массива возрастов с~той же позицией, мы получим парный элемент для~текущего имени.

\cppfile[firstline=28, lastline=46]{projects/07/name_age.cpp}

\noindent Отметим, что в~исходных данных может быть несколько пар с~одинаковыми именами. Мы сохраним их все, соблюдая тот порядок, в~котором они были исходно (\emph{стабильная сортировка}).

Вспомогательная функция поиска элемента в~массиве, начиная с~указанной позиции:

\cppfile[firstline=18, lastline=26]{projects/07/name_age.cpp}



%%==============================================
\paragraph{Функция \texttt{main} и ввод данных.}
%%==============================================
Тело \code{main} может выглядеть примерно таким образом:

\cppfile[firstline=81, lastline=81]{projects/07/name_age.cpp}
\cppfile[firstline=83, lastline=97]{projects/07/name_age.cpp}

Мы разместили каждое логическое действие в~отдельной функции. Самое сложное из~ещё не~рассмотренных нами действий "--- это ввод данных от~пользователя.

\cppfile[firstline=58, lastline=73]{projects/07/name_age.cpp}

Пары будут считываться, до~тех пор, пока мы не~завершим ввод при~помощи \code{Ctrl+D} в~\name{Linux} или \code{Ctrl+Z} в~\name{Windows}. Или пока не~введём некорректное значение для~возраста (почему для~возраста?), например, символ \code{'|'}. Появится ли некорректная пара в~последнем случае?

Вывод же на~печать пар имя/возраст теперь совсем не~составит труда.

\cppfile[firstline=75, lastline=79]{projects/07/name_age.cpp}



%%=======================
\paragraph{Тестирование.}
%%=======================
Воспользуйтесь перенаправлением ввода из~файла, чтобы протестировать полученную программу. Заготовьте пары в~отдельном файле \code{name\_age\_test\_in.txt}, и подайте их на~ввод программе, как это было показано в~предыдущем разделе на~странице~\pageref{par:inout}.



%%================
\paragraph{P.\,S.}
%%================
Слегка доработайте программу, улучшив приветствие пользователю и добавив <<штатное>> завершение ввода, например, по~команде \code{End}.



%%================
\WhatToReadSection
%%================
\textcite{Stroustrup:2016:ru}: \textbf{глава~9}



%%===============
\ExercisesSection
%%===============
\begin{exercise}
\item Выполните упражнения из~\textbookref{главы~8} учебника.

\end{exercise}
