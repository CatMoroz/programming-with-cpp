% !TEX encoding   = UTF8
% !TEX spellcheck = ru_RU
% !TEX root = seminars.tex

%%================
\chapter{Введение}
%%================

%%=====================
\section{Цели и задачи}
%%=====================
\textit{Усвоить основы разработки, тестирования и отладки программ, базовые конструкции языка \lang{C++} и элементы стандартной библиотеки и получить практические навыки их использования при~решении ряда простых и более сложных задач.}

%%==========================
\paragraph{Структура курса.}
%%==========================
\begin{itemfeature}
  \item О курсе в~целом.
  \item Взаимосвязь с~другими курсами.

  \begin{flushleft}\hspace{-4em}\begin{tikzpicture}[node font=\small, >=Stealth]
    \graph [layered layout, components go right top aligned, layer sep=1em]
    {
      AppliedTasks/{Прикладные задачи} [blue, draw=blue, dashed, rounded corners] // [layered layout] {"Аэродинамика", "Динамика полёта", "Прочность", "Проектирование"};

      AppliedSoftware/{Прикладное программное обеспечение} [blue, draw=blue, dashed, rounded corners] // [layered layout, edge=white] {
        OpenFoam, SU2, Gmsh, FreeCAD, ParaView
      };

      OS/{Операционные системы} [blue, draw=blue, dashed, rounded corners, edge=black] // [tree layout] {
        Unix -> {
          Linux -> Android, FreeBSD -> {MacOS, iOS}, Solaris;
        },
        Windows
      };

      ComputerArchitecture/{Архитектура компьютера} [blue, draw=blue, dashed, rounded corners] // [layered layout, edge=<-] {
        "Архитектура и язык ассемблера"
        -> "Микроархитектура"
        -> "<<железо>>"
      };

      ComputerLanguages/{Языки программирования} [blue, draw=blue, dashed, rounded corners] // [layered layout, components go right top aligned, edge={<-, white}] {
        {Python, Java, PHP, HTML, Perl}
        -> {C++, Rust, ObjectiveC[as={Objective C}], Fortran}
        -> {C, Forth, Pascal}
      };

      Algorithms/{Алгоритмы и структуры данных} [blue, draw=blue, dashed, rounded corners] // [layered layout, edge=white] {
        {"Двоичный поиск", "Деревья"}
        -> {"Сортировка", "Хэш-таблицы"}
        -> {"Волновой алгоритм", "Графы"}
      };

      AppliedTasks ->[white] AppliedSoftware ->[white] {OS, ComputerLanguages} ->[white] {ComputerArchitecture, Algorithms};
    };
  \end{tikzpicture}\end{flushleft}

  \item Контрольно-проверочные мероприятия.
\end{itemfeature}



%%===========================
\section{Основная литература}
%%===========================
\cite{Stroustrup:2016:ru}

\nocite{Kernighan:2004:ru, Meyers:2006:ru, Meyers:2000:ru, Meyers:2002:ru, Meyers:2016:ru, Josuttis:2014:ru, Stroustrup:2006:ru, Stroustrup:2013:en}




%%===========================
\section{Материалы и задания}
%%===========================
Значительная часть обучающих упражнений и заданий содержится в~книге Бьярне Страуструпа. Дополнительные материалы размещены на~\href{\yadiskurl}{яндекс-диске\footnote{Материалы на Яндекс-диске: \nolinkurl{\yadiskurl}}}:
\begin{itemfeature}
  \item \codebf{books} "--- основная и дополнительная литература;
  \item \codebf{cpp-lectures} "--- лекционные материалы;
  \item \codebf{cpp-seminars} "--- семинарские материалы;
  \begin{itemize}
  	\item \codebf{/libs} "--- библиотеки, которые используются на занятиях;
  	\item \codebf{/program.pdf} "--- программа курса (темы для беседы на зачёте);
  	\item \codebf{/progress.pdf} "--- планы, успеваемость и проверочные мероприятия;
  	\item \codebf{/seminars.pdf} "--- вспомогательная методичка по материалам занятий.
  \end{itemize}
\end{itemfeature}

Исходный код примеров из данного учебного пособия доступен в \href{\courseselfurl}{репозитории}\footnote{Репозиторий данного пособия: \nolinkurl{\courseselfurl}} на~Git\,Hub, каталог \code{projects}.



%%===============================
\section{Программное обеспечение}
%%===============================
Интегрированная среда разработки (IDE "--- \textenglish{Integrated Development Environment}):
\begin{itemfeature}
  \item текстовый редактор,
  \item компилятор языка \lang{C++} (с~поддержкой стандарта \lang{C++14} или выше),
  \item редактор связей,
  \item средства сборки,
  \item средства отладки.
\end{itemfeature}



%%===========================================
\section{Установка и настройка рабочей среды}\label{sect:workEnv}
%%===========================================
Мы настоятельно рекомендуем в~качестве IDE установить \href{\vscodeurl}{\name{VS\,Code}}\footnote{\textenglish{Microsoft VS\,Code:} \nolinkurl{\vscodeurl}} от~Microsoft. Эта программа активно развивается, имеет полноценный современный редактор кода и поддерживает интеграцию со~множеством полезных инструментов (системой контроля версий, средствами форматирования кода, консолью). Распространяется свободно, кроссплатформенная, то есть доступна не~только под~Windows, но и под~Unix: GNU/Linux, Mac\,OS\,X. А также доступна на~территории России.

Запустите \name{VS\,Code}. Перейдите во вкладку <<Расширения>> на панели слева. Установите плагин \textenglish{C/C++ for Visual Studio Code}.

Далее необходимо установить \href{\smartgiturl}{\name{Smartgit}}\footnote{Syntevo Smartgit: \nolinkurl{\smartgiturl}} (или, в крайнем случае, просто \href{\giturl}{\git}\footnote{Система контроля версий Git: \nolinkurl{\giturl}}), оставляя рекомендуемые (по умолчанию) параметры в~тех местах, где вы сомневаетесь, что выбрать. \name{Smartgit} включает в себя дистрибутив \git-а, поэтому последний не нужно ставить отдельно.

\name{Smartgit} "--- это удобный и мощный графический пользовательский интерфейс, то есть окошки и кнопочки, для работы с~системой контроля версий \git. Это коммерческий продукт, однако, разработчик предоставляет некоммерческую лицензию для академических заведений, которую можно запросить, заполнив \href{\smartgitacademicurl}{форму}\footnote{Запрос лицензии: \nolinkurl{\smartgitacademicurl}} на сайте.

В~сочетании с дополнительными плагинами, \name{VS\,Code} частично предоставляет подобные возможности. Но, увы, это пока не так удобно.

Система контроля версий "--- неотъемлемый инструмент разработки сложных программ. Также он весьма полезен при~разработке небольших программ.



%%=====================
\paragraph{ОС Windows.}
%%=====================
Совместно с~\name{VS\,Code} используйте \href{\mingwurl}{\name{MinGW-w64}}\footnote{MinGW-w64: \nolinkurl{\mingwurl}} (компилятор, компоновщик, отладчик и прочее).

Для~этого:
\begin{itemfeature}
	\item \href{\sevenzipurl}{Распакуйте}\footnote{Архиватор 7-Zip: \nolinkurl{\sevenzipurl}} архив \name{MinGW-w64} в~каталог \code{c:\backslash{}mingw-64} или любой другой, куда вы обычно устанавливаете программы.

	\item \href{\addtosyspathurl}{Добавьте}\footnote{Adding Path to MinGW: \nolinkurl{\addtosyspathurl}} путь к~компилятору (\code{bin\backslash{}g++.exe}) в~список стандартных системных путей (системная переменная \name{PATH}).
\end{itemfeature}

Удобная Unix-подобная командная среда \name{Git\,Bash} идёт в~комплекте с~системой контроля версий \git{} (а также в~комплекте со~\name{Smartgit}-ом). Простые программы мы будем собирать непосредственно в~командной среде. Помимо прочего, она облегчит тестирование программ в~автоматическом и полуавтоматическом режимах.

\href{\embedgitbashurl}{Добавьте}\footnote{\textenglish{VS Code --- Integrate Git Bash as Default Terminal}: \nolinkurl{\embedgitbashurl}} \name{Git\,Bash} в~список доступных терминалов \name{VS\,Code} и выберите его в~качестве терминала по~умолчанию. (Файл с~именем \code{bash.exe} нужно найти в~каталоге установки \name{Smartgit}-а.)



%%==================
\paragraph{ОС Unix.}
%%==================
Пользователи Unix и подобных ей операционных систем могут установить средства сборки программ \name{GNU}/\GCC{} при~помощи менеджера системы управления пакетами (например, \code{apt} в~\name{Ubuntu} или \code{pamac} в~\name{Manjaro}). Командная среда \code{bash}, или подобная ей \code{zsh}, обычно доступна из~коробки и не~требует отдельной установки.



%%===================================
\paragraph{Если что-то пошло не так,}
%%===================================
повторите процедуру \textbf{спокойно}, выполняя \textbf{в~точности} все указанные выше действия. Попробуйте использовать только латиницу для~каталогов установки и проектов, иногда русские буквы в~путях или пробелы могут вызывать ошибки.



%%=================================
\paragraph{И теперь не получилось?}
%%=================================
Хм-м... Тогда обратитесь за~помощью к~товарищам в~группе, к~студентам старших курсов или, в~конце концов, к~преподавателю.



%%==================
\section{Что читать}
%%==================
\textcite{Stroustrup:2016:ru}: \textbf{главы~0, 1 и~2}
