% !TEX encoding   = UTF8
% !TEX spellcheck = ru_RU
% !TEX root = seminars.tex

%%==================
\chapter{Что дальше}
%%==================
Станете ли вы профессиональным программистом или экспертом по~языку \lang{С++}, прочитав эту книгу?\footcite[Текст взят из~подраздела 0.1.3]{Stroustrup:2016:ru} Конечно, нет! Настоящее программирование "--- это тонкое, глубокое и очень сложное искусство, требующее знаний и технических навыков. Рассчитывать на~то, что за~четыре месяца вы станете экспертом по~программированию, можно с~таким же успехом, как и на~то, что за~полгода или даже год вы полностью изучите биологию, математику или иностранный язык (например, китайский, английский или датский) или научитесь играть на~виолончели. Если подходить к~изучению книги серьёзно, то можно ожидать, что вы сможете писать простые полезные программы, читать более сложные программы и получите хорошие теоретическую и практическую основы для~дальнейшей работы.

Прослушав этот курс, лучше всего поработать над~реальным проектом. Ещё лучше параллельно с~работой над~реальным проектом приступить к~чтению какой-нибудь книги профессионального уровня (например, \textcite{Stroustrup:2013:en}), более специализированной книги, связанной с~вашим проектом, например, документации по~библиотеке \name{Qt} для~разработки графического пользовательского интерфейса (GUI) или справочника по~библиотеке \name{ACE} для~параллельного программирования, или учебника, посвященного конкретному аспекту языка \lang{С++}, например \textcite{Koenig:2002:ru, Sutter:2008:ru, Gamma:2004:ru}.

В~конечном итоге вам придётся приступить к~изучению некоторого другого языка программирования. Невозможно стать профессионалом в~области программного обеспечения (даже если программирование не~является вашей основной специальностью), зная только один язык программирования.
