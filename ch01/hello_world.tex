% !TEX encoding   = UTF8
% !TEX spellcheck = ru_RU
% !TEX root = ../seminars.tex

%%=====================
\chapter{Hello, World!}
%%=====================

%%==================================================
\section{Архитектура Фон-Неймана, уровни абстракции}
%%==================================================
Схематическое устройство компьютера. Алгоритмы и языки программирования. Абстракция от~реального <<железа>>.



%%=================
\section{Программы}
%%=================
\lang{С++}\footcite{Stroustrup:2019:ru} является компилируемым языком. Для работы программы её исходный текст должен быть обработан с~помощью компилятора, который создаёт объектные файлы, объединяемые компоновщиком в~выполнимую программу. Обычно программы на~языке \lang{С++} состоят из~многих файлов с~исходными текстами (именуемыми просто \textit{исходными файлами}).

\begin{center}\begin{tikzpicture}[node font=\small, >=Stealth, line width=1pt]
    \graph [grow right sep, left anchor=east, right anchor=west, nodes=draw] {
    "файл1.cpp" -> c1/{компиляция} [ellipse] -> "файл1.o" ->[right anchor=north west] "компоновка" [ellipse, yshift=-2.5ex] -> "выполнимый файл"[yshift=-2.5ex];
    "файл2.cpp" -> c2/{компиляция} [ellipse] -> "файл2.o" ->[right anchor=south west] "компоновка";
    };
\end{tikzpicture}\end{center}

Выполнимая программа создаётся для~определённой комбинации аппаратного обеспечения и операционной системы; её нельзя просто перенести, скажем, из~компьютера Мac в~компьютер с~Windows. Говоря о~переносимости программ \lang{С++}, мы обычно имеем в~виду переносимость исходного кода, т.\,е. исходный код может быть успешно скомпилирован и выполняться в~разных
системах.

Стандарт ISO~\lang{С++} определяет два типа сущностей.
\begin{itemize}
\item \textit{Фундаментальные возможности языка}, такие как встроенные типы (например, \code{char} и \code{int}) или циклы (например, инструкции \code{for} и \code{while}).

\item \textit{Компоненты стандартных библиотек}, такие как контейнеры (например, \code{vector}, и \code{map}) или операции ввода--вывода (например, \code{<<} и \code{getline()}).
\end{itemize}

Компоненты стандартной библиотеки представляют собой совершенно обычный код \code{С++}, предоставляемый каждой реализацией языка. То есть стандартная библиотека \code{С++} может быть реализована в~самом \code{С++} (и реализуется "--- с~очень небольшим использованием машинного кода для~таких вещей, как переключение контекста потока). Это означает, что \code{С++} достаточно выразителен и эффективен для~самых сложных задач системного программирования.

\code{С++} является статически типизированным языком, т.е. тип каждой сущности (например, объекта, значения, имени или выражения) должен быть известен компилятору в~точке использования. Тип объекта определяет набор применимых к~нему операций.



%%===========================
\section{Структура каталогов}
%%===========================
Настоятельно рекомендуем упорядочивать файлы. Например, для семинарских программ и проектов можно использовать следующую структуру:
\begin{flushleft}\begin{tikzpicture}[node font=\small\ttfamily]
		\matrix[row sep=0.2ex, column sep=0.5em]
		{
			\node (root) {projects/}; &  & &[-5em] \node[right] {\rmfamily все программы и проекты}; \\
			& \node[right] (sem1) {01/}; & & \node[right] {\rmfamily первое занятие}; \\
			& & \node[right] (hello) {hello.cpp}; & \\
			& \node[right] (sem2) {02/}; & & \node[right] {\rmfamily второе занятие}; \\
			& & \node[right] (test1) {empty.txt}; & \\
			& & \node[right] (lsm) {least\_squares.cpp}; & \\
			& & \node[right] (test2) {line\_approx.txt}; & \\
			& & \node[right] (test3) {line\_exact.txt}; & \\
			& & \node[right] (plot) {plot.py}; & \\
			& \node[right] {\ldots}; & & \\
			& \node[right] (bin) {bin/}; & & \node[right] {\rmfamily исполняемые файлы}; \\
			& \node[right] (lib) {lib/}; & & \node[right] {\rmfamily общие файлы и библиотеки}; \\
			& & \node[right] (graphlib) {Graph\_lib/}; & \\
			& & \node[right] (std) {std\_lib\_facilities.h}; & \\
			& & \node[right] {\ldots}; & \\
			& \node[right] (task1) {task-1/}; & & \\
			& \node[right] (task2) {task-2/}; & & \\
			& \node[right, text width=4em] (git) {.git/}; & & \node[right] {\rmfamily репозиторий системы контроля версий}; \\
			& \node[right] (format) {.clang-format}; & & \node[right] {\rmfamily стиль оформления кода}; \\
			& \node[right] (ignore) {.gitignore}; & & \node[right] {\rmfamily исключения для системы контроля версий}; \\
		};

		\draw (root) |- (sem1) (sem1) |- (hello);
		\draw (root) |- (sem2) (sem2) |- (test1)
		                       (sem2) |- (lsm)
		                       (sem2) |- (test2)
		                       (sem2) |- (test3)
		                       (sem2) |- (plot);
		\draw (root) |- (bin);
		\draw (root) |- (lib) (lib) |- (graphlib)
		                      (lib) |- (std);
		\draw (root) |- (task1);
		\draw (root) |- (task2);
		\draw (root) |- (git);
		\draw (root) |- (format);
		\draw (root) |- (ignore);
\end{tikzpicture}\end{flushleft}



%%=====================================
\section{Классическая первая программа}
%%=====================================
Наберите код классической первой программы (\code{hello.cpp}):

\cppfile{projects/01/hello.cpp}

Чтобы скомпилировать код, необходимо перейти (команда \code{cd}) в~каталог всех проектов:

\console'$ cd /path/to/your/projects'%$

\noindent и уже оттуда выполнить команду:

\console'$ g++ -Og -Wall -Wextra -pedantic -Ilib -o p 01/hello.cpp'%$

Здесь мы использовали следующие ключики (флаги) компилятора \code{g++}:
\begin{center}\begin{tabular}{lp{0.8\textwidth}}
    \toprule
    \code{-Og} & Задаёт разумный уровень оптимизации при~сохранении скорости компиляции и возможности отладки \\
    \code{-Wall} & Включает все предупреждения относительно конструкций, которые многие пользователи считают сомнительными и которые легко исправить \\
    \code{-Wextra} & Включает дополнительные предупреждения, которые не~включает \code{-Wall} \\
    \code{-pedantic} & Сообщает обо~всех нарушениях стандарта языка \lang{C++}. Отвергает программы, которые используют запрещённые расширения \\
    \code{-Ilib} & Добавляет каталог \code{lib} в~стандартные пути поиска компилятора. Таким образом, компилятор находит наш заголовок \code{std\_lib\_facilities.h} в~стандартных путях (когда, используя директиву \code{\#include}, мы указываем имя файла в~угловых скобках, а не в~двойных кавычках) \\
    \code{-o p} & Даёт исполняемому файлу имя \code{p}. В~\name{Windows} к~имени автоматически добавляется расширение \code{exe} \\
    \bottomrule
\end{tabular}\end{center}

Следом за~ключиками мы перечисляем все файлы с~исходным кодом (только \code{*.cpp} файлы). На первых порах у~нас будут программы, в которых весь код размещается в~одном файле. Немного позже мы начнём распределять код на~несколько модулей.

Если компиляция прошла успешно, то на~экране не~появится никаких сообщений. Можно вывести список файлов в~текущем каталоге, чтобы убедиться, что исполняемый файл с~именем \code{p} действительно появился.

\begin{consolecode}
$ ls
01  02  lib  p
\end{consolecode}

Теперь можно запустить программу на исполнение:
\begin{consolecode}
$ ./p
Hello, World!
\end{consolecode}

Текущий каталог (обозначается символом <<\code{.}>>) не~входит в~системные пути Unix, поэтому мы обязаны написать \code{./p}. Напротив, утилита \code{ls} расположена в~системных путях, поэтому не~требует уточнения пути.

В~дальнейшем рекомендуем сохранять бинарный (исполняемый) файл в~каталоге \code{bin}:
\begin{consolecode}
$ rm p
$ mkdir bin
$ ls
01  02  bin  lib
\end{consolecode}

\noindent Мы удалили (команда \code{rm}) файл~\code{p} и затем создали (команда \code{mkdir}) каталог \code{bin}.

\begin{consolecode}
$ g++ -Og -Wall -Wextra -pedantic -Ilib -o bin/p 01/hello.cpp
$ bin/p
Hello, World!
\end{consolecode}

\emph{Совет: чтобы избежать повторного набора команд в терминале, используйте стрелки~\code{\uparrow} и~\code{\downarrow} для~навигации по~истории команд.}



%%=============================
\section{Редактирование текста}
%%=============================
Набор исходного текста "--- неотъемлемая часть разработки программ. Используйте <<горячие>> клавиши для~более быстрого редактирования. Краткий список наиболее популярных комбинаций, поддерживаемых даже простыми редакторами:

{ % hot keys
\newcommand*{\hotkey}[1]{\fbox{\texttt{\small #1}}}
\newcommand*{\hotplus}{{\small\,+\,}}
\newcommand*{\hotkeys}[2]{\hotkey{#1}\hotplus\hotkey{#2}}
\newcommand*{\hotkeyss}[3]{\hotkey{#1}\hotplus\hotkey{#2}\hotplus\hotkey{#3}}

\begin{longtable}[l]{@{}rp{0.7\textwidth}@{}}
    \endhead
    \endfoot

    \hotkeys{Ctrl}{A} & Выделить всё \\[0.5em]

    \hotkeys{Ctrl}{C} & Скопировать выделение \\
    \hotkeys{Ctrl}{V} & Вставить \\[0.5em]

    \hotkeys{Ctrl}{Z} & Отменить последнее действие \\[0.5em]

    \hotkey{Home} & Переместить курсор в~начало\slash{}конец текущей строки \\
    \hotkey{End}  & \\[0.5em]

    \hotkeys{Shift}{Home} & Выделить символы с~текущей позиции и до~начала\slash{}конца строки \\
    \hotkeys{Shift}{End}  & \\[0.5em]

    \hotkeys{Ctrl}{Home} & Переместить курсор в~начало\slash{}конец файла \\
    \hotkeys{Ctrl}{End}  & \\
\end{longtable}

\noindent Дополнительные полезные комбинации в~\name{VS\,Code}:
\begin{longtable}[l]{@{}rp{0.7\textwidth}@{}}
    \endhead
    \endfoot

    \hotkeys{Ctrl}{\slash} & Комментировать\slash{}раскомментировать текущую строку или выделение \\[0.5em]

    \hotkeys{Alt}{\uparrow}   & Сместить текущую строку или выделение вверх\slash{}вниз \\
    \hotkeys{Alt}{\downarrow} & \\[0.5em]

    \hotkeyss{Alt}{Shift}{\uparrow}   & Дублировать текущую строку или выделение вверх\slash{}вниз \\
    \hotkeyss{Alt}{Shift}{\downarrow} & \\[0.5em]

    \hotkeyss{Ctrl}{Alt}{\uparrow}   & Редактировать столбец \\
    \hotkeyss{Ctrl}{Alt}{\downarrow} & \\
\end{longtable}
} % hot keys

Полезно научиться набирать текст вслепую, то есть не~глядя на~клавиатуру. В~сочетании с~использованием горячих клавиш, это позволяет достичь существенного ускорения при~наборе исходного кода. Таким образом, остаётся больше времени для~размышлений над~структурой и логикой самой программы. В~качестве примера он-лайн клавиатурного тренажёра приведём \href{\typingtutorurl}{эту}\footnote{Он-лайн клавиатурный тренажёр: \nolinkurl{\typingtutorurl}} ссылку.

Используйте моноширинный шрифт для~кода, например, \name{JetBrains\,Mono}. По~желанию, его можно \href{\jetbrainsfonturl}{скачать}\footnote{Шрифты JetBrains Mono: \nolinkurl{\jetbrainsfonturl}} и \href{\setupfonturl}{установить}\footnote{Как установить шрифт в~систему: \nolinkurl{\setupfonturl}} в~систему. Чтобы указать шрифт редактору, откройте файл с~настройками \name{VS\,Code} и добавьте такие строки:
\begin{minted}[fontsize=\small]{json}
    "editor.fontFamily": "JetBrains Mono",
    "editor.fontLigatures": false
\end{minted}
\noindent Если включить лигатуры, то некоторые составные операторы будут заменяться на более привычное начертание. Например, оператор \code{!=} будет выглядеть как \(\neq\).

Аккуратно оформляйте код, соблюдая правила хорошего \href{\codestyleurl}{стиля}\footnote{Стиль кода: \nolinkurl{\codestyleurl}}. Современные средства позволяют автоматически форматировать код согласно заданному стилю. Описание стиля \code{projects/.clang-format} есть в~\href{\courseselfurl}{репозитории}. В~настройки добавьте:

\begin{minted}[fontsize=\small]{json}
    "editor.formatOnSave": true
\end{minted}

Можно изменить стиль по вашему вкусу. Для этого отредактируйте загруженный файл, используя \href{\clangformatdocurl}{документацию}\footnote{Опции Clang-format: \nolinkurl{\clangformatdocurl}} утилиты \name{Clang-format}.



%%====================
\section{Unix утилиты}\label{sect:utils}
%%====================
Совместно с~командной средой \code{bash} (\textenglish{\textbf{b}ourne \textbf{a}gain \textbf{sh}ell}) поставляется широкий набор полезных программ, или утилит (\textenglish{utilities}).

\console/$ cd dir/%$

\textenglish{\textbf{c}hange \textbf{d}irectory} "--- перейти в~каталог \code{dir}. Если каталог опущен, то перейти в~домашний каталог. (Определяется значением переменной среды \code{HOME}.) Файловая система \textenglish{Unix} представляет собой единое дерево, и любой абсолютный путь начинается с~корня (\code{/}). Точка (\code{.}) означает текущий каталог, две точки (\code{..}) "--- родительский каталог.

\console/$ cd ../%$

Перейти в~родительский каталог.

\console|$ cd /|%$

Перейти в~корневой каталог.

\console/$ pwd/%$

\textenglish{\textbf{p}rint \textbf{w}orking \textbf{d}irectory} "--- вывести на~экран абсолютный путь к~текущему каталогу.

\console/$ ls dir/%$

\textenglish{\textbf{l}i\textbf{s}t} "--- вывести на~экран содержимое каталога \code{dir}. Если каталог опущен, то по~умолчанию используется текущий каталог.

\console/$ man cmd/%$

\textenglish{\textbf{man}ual} "--- вывести на~экран подробную справку о~команде \code{cmd}.

\console/$ cmd --help/%$

Вывести на~экран короткую справку о~команде \code{cmd}. Полезно, если утилита \code{man} недоступна.

\console/$ mkdir dir/%$

\textenglish{\textbf{m}a\textbf{k}e \textbf{dir}ectory} "--- создать каталог \code{dir}.

\console/$ rmdir dir/%$

\textenglish{\textbf{r}e\textbf{m}ove \textbf{dir}ectory} "--- удалить пустой каталог \code{dir}.

\console/$ rm file/%$

\textenglish{\textbf{r}e\textbf{m}ove} "--- удалить файл. Используя опцию \code{-r} можно рекурсивно удалить непустой каталог.

\console/$ mv src dst/%$

\textenglish{\textbf{m}o\textbf{v}e} "--- перемещает/переименовывает файл \code{src} в~\code{dst}. Имя \code{dst} может быть каталогом, тогда \code{mv} перемещает \code{src} туда.

\console/$ cat file.../%$

\textenglish{\textbf{cat}enate} "--- связывает файлы и выводит объединённое содержимое на~экран.

\console/$ less file/%$

Более мощный аналог утилиты \code{more} "--- постраничного фильтра для~просмотра файлов. Часто применяется для~буферизации вывода на~экран, используя конвейер:

\console/$ odjdump -d prog | less/%$


\console/$ vim file/%$

\textenglish{\textbf{v}isual editor \textbf{im}proved} "--- открыть файл для~редактирования. Пройти вводный курс по~использованию этого мощного редактора можно запустив команду:

\console/$ vimtutor/%$



%%==================================
\section{Интерпретатор \name{shell}}\label{sect:shell}
%%==================================
Когда система (в~терминале) выдаёт приглашение \code{\$} и вы вводите команды для~выполнения, вы имеете дело не с~ядром самой системы, а с~неким посредником, называемым интерпретатором команд, или \name{shell}\footnote{Более подробно об~интерпретаторе \name{shell} и других возможностях системы Unix излагается, например, в~книге \fullcite{Kernighan:1992:ru}}. Это обычная программа, но она может делать удивительные вещи. Применение программы--посредника обеспечивает три главных преимущества:

\begin{itemfeature}[itemsep=\baselineskip]
    \item Сокращённые имена файлов: можно задать целое множество файлов в~качестве аргументов команде, указав шаблон для имён: \name{shell} будет искать файлы, имена которых соответствуют заданному шаблону.

    \console/$ ls *.cpp *.c/%$

    Вывести на~печать имена всех файлов в~текущем каталоге, которые оканчиваются на~\code{.cpp} или \code{.c}. Символ \code{*} в~шаблоне соответствует любой последовательности символов. Поддерживаются и другие специальные символы для~задания шаблона.


    \item Переключение ввода--вывода: вывод любой программы можно направить в~файл, а не на~терминал, ввод можно получать из~файла, а не с~терминала. Ввод и вывод можно даже передать другим программам.

    \console/$ ls *.cpp > cppfiles.txt/%$

    Имена всех файлов, оканчивающихся на~\code{.cpp}, направить в~файл \code{cppfiles.txt}. Файл будет создан, если не~существует, или перезаписан, если существует.

    \console/$ wc -l < main.cpp/%$

    Подсчитать количество строк (опция \code{-l}) в~файле \code{main.cpp}.

    \console/$ cat main.cpp | wc -l/%$

    То же. Символ~\code{|} обозначает конвейер. Вывод команды слева передаётся на~ввод команде справа. Можно организовывать в~конвейере цепочку любой длины.

    \console/$ wc -l main.cpp/%$

    То же, используя лишь возможности самой утилиты \code{wc} (\textenglish{\textbf{w}ord \textbf{c}ount}).


    \item Создание собственной среды: можно определить свои собственные команды и правила сокращений.
\end{itemfeature}



%%============================================
\section{Система контроля версий \texttt{Git}}
%%============================================
Основы работы в~\git{} изложены в~книге \cite{Chacon:2023:ru}.

Это довольно обширный материал и представляет собой отдельный курс. Поэтому здесь мы рекомендуем прочитать главы <<Введение>> и <<Основы \git{}>> для первого знакомства с системой контроля версий.

Затем, полезно получить некоторой опыт работы с данной системой. И далее, перечитать эти главы, а также прочитать главу <<Ветвление в~\git{}>>. Этих возможностей вам будет вполне достаточно для успешного освоения данного курса.

Поняв суть и возможности системы контроля версий \git{}, очень просто освоить \name{Smartgit} "--- графическую обёртку. Каждая кнопка интерфейса выполняет в~консоли одну или несколько команд \git{}-а. Эти команды печатаются в~окне \textenglish{Output}. Попутно в~разных окнах отображаются состояние файлов, изменения от~версии к~версии, а также полное дерево истории коммитов. Встроенные средства позволяют не только сравнивать, принимать/отклонять, но и редактировать изменения.



%%================
\WhatToReadSection
%%================
\textcite{Stroustrup:2016:ru}: \textbf{глава~3} \\\indent
\textcite{Chacon:2023:ru}: \textbf{глава <<Основы \git{}>>}



%%===============
\ExercisesSection
%%===============
\begin{exercise}
\item Напишите программу, которая по заданному \(n\) находит \(n\)-е число Фибоначчи \(F_n\). Числа Фибоначчи определяются соотношениями:
\[
F_n = F_{n-1} + F_{n-2},\quad F_1 = F_2 = 1.
\]
Необходимо ли хранить в памяти все вычисленные числа Фибоначчи?

\item Напишите программу, которая принимает на входе число и печатает на экран его зеркальную копию, то есть, в <<перевёрнутом>> виде.

\item Напишите функцию, которая принимает на входе целое число и возвращает ответ, является ли оно простым. Протестируйте работу вашей функции.

\item Вычислите и сравните левую и правую части тождеств:
\begin{flushleft}
    \begin{enumerate*}[label=\arabic*), itemjoin={\qquad}]
        \item \(\sum\limits_{k=0}^{\infty}\dfrac{1}{(2k)!} = \dfrac{e + e^{-1}}{2}\);
        \item \(\prod\limits_{k=1}^{\infty} \dfrac{4k^2}{4k^2 - 1} = \dfrac{\pi}{2}\);
        \item \(e = \lim\limits_{n\to\infty} \dfrac{n}{\sqrt[n]{n!}}\);
    \end{enumerate*}
    \begin{flushright}
        \begin{enumerate*}[label=\arabic*), itemjoin={;\qquad}, resume]
            \item \(\dfrac{2}{1}\cdot
            \left( \dfrac{4}{3} \right)^{\frac{1}{2}}\cdot
            \left( \dfrac{6}{5}\dfrac{8}{7} \right)^{\frac{1}{4}}\cdot
            \left( \dfrac{10}{9}\dfrac{12}{11}\dfrac{14}{13}\dfrac{16}{15} \right)^{\frac{1}{8}}\cdot
            \ldots = e\).
        \end{enumerate*}
    \end{flushright}
\end{flushleft}
\end{exercise}