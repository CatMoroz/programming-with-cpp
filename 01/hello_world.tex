% !TEX encoding   = UTF8
% !TEX spellcheck = ru_RU
% !TEX root = ../../seminars.tex

%%=====================
\chapter{Hello, World!}\label{chap:helloworld}
%%=====================

%%==================================================
\section{Архитектура Фон-Неймана, уровни абстракции}
%%==================================================
Схематическое устройство компьютера. Алгоритмы и языки программирования. Абстракция от реального <<железа>>.



%%=================
\section{Программы}
%%=================
\lang{С++}\footcite{Stroustrup:2019:ru} является компилируемым языком. Для работы программы её исходный текст должен быть обработан с помощью компилятора, который создаёт объектные файлы, объединяемые компоновщиком в выполнимую программу. Обычно программы на языке \lang{С++} состоят из многих файлов с исходными текстами (именуемыми просто \textit{исходными файлами}).

\begin{center}\begin{tikzpicture}[node font=\small, >=Stealth, line width=1pt]
  \graph [grow right sep, left anchor=east, right anchor=west, nodes=draw] {
  "файл1.cpp" -> c1/{компиляция} [ellipse] -> "файл1.o" ->[right anchor=north west] "компоновка" [ellipse, yshift=-2.5ex] -> "выполнимый файл"[yshift=-2.5ex];
  "файл2.cpp" -> c2/{компиляция} [ellipse] -> "файл2.o" ->[right anchor=south west] "компоновка";
  };
\end{tikzpicture}\end{center}

Выполнимая программа создаётся для определённой комбинации аппаратного обеспечения и операционной системы; её нельзя просто перенести, скажем, из компьютера \name{Мac} в компьютер с \name{Windows}. Говоря о переносимости программ \lang{С++}, мы обычно имеем в виду переносимость исходного кода, т.\,е. исходный код может быть успешно скомпилирован и выполняться в разных
системах.

Стандарт \name{ISO} \lang{С++} определяет два типа сущностей.
\begin{itemize}
\item \textit{Фундаментальные возможности языка}, такие как встроенные типы (например, \code{char} и \code{int}) или циклы (например, инструкции \code{for} и \code{while}).

\item \textit{Компоненты стандартных библиотек}, такие как контейнеры (например, \code{vector}, и \code{map}) или операции ввода--вывода (например, \code{<<} и \code{getline()}).
\end{itemize}

Компоненты стандартной библиотеки представляют собой совершенно обычный код \code{С++}, предоставляемый каждой реализацией языка. То есть стандартная библиотека \code{С++} может быть реализована в самом \code{С++} (и реализуется "--- с очень небольшим использованием машинного кода для таких вещей, как переключение контекста потока). Это означает, что \code{С++} достаточно выразителен и эффективен для самых сложных задач системного программирования.

\code{С++} является статически типизированным языком, т.е. тип каждой сущности (например, объекта, значения, имени или выражения) должен быть известен компилятору в точке использования. Тип объекта определяет набор применимых к нему операций.



%%==================================
\section{Метод наименьших квадратов}
%%==================================

%%===============================
\subparagraph{Постановка задачи.}
%%===============================
Рассмотрим регрессию (зависимость) следующего вида:
\begingroup
\newcommand{\SumN}{\ensuremath{\sum\limits_{i=1}^{N}}}
\[
  y_i = a + b x_i + \varepsilon_i,\quad i = \overline{1, N}.
\]
Среди всевозможных значений \(\{ a, b \}\) будем искать такие, которые приводят к минимальной сумме квадратов отклонений (ошибок):
\[
  S_{\varepsilon} = \SumN \varepsilon_i^2 = \SumN (y_i - a - b x_i)^2\quad\rightarrow\quad \min\limits_{a, b}.
\]
Запишем условие существования экстремума:
\[
\begin{array}{l}
  \dfrac{\partial}{\partial a} S_{\varepsilon} = -2 \SumN (y_i - a - b x_i) = 0, \\[2ex]
  \dfrac{\partial}{\partial b} S_{\varepsilon} = -2 \SumN (y_i - a - b x_i) x_i = 0, \\
\end{array}
\]
откуда получим систему линейных уравнений:
\[
\left\{ \begin{array}{l}
        \SumN y_i - N a - b\SumN x_i = 0, \\[2ex]
        \SumN x_i y_i - a\SumN x_i - b\SumN x_i^2 = 0. \\
        \end{array} \right.
\]
Вводя обозначение для среднего арифметического множества значений некоторой величины \(f\):
\[
  \bar f = \dfrac{1}{N}\SumN f_i,
\]
перепишем систему в виде:
\[
\left\{ \begin{array}{l}
        \bar y - a - b\bar x = 0, \\
        \overline{x y} - a\bar x - b\overline{x^2} = 0, \\
        \end{array} \right.
\]
и получим искомое решение:
\[
\boxed{\begin{array}{l}
       a = \bar y - b\bar x, \\
       b = \dfrac{\overline{x y} - \bar x\bar y}{\overline{x^2} - \bar x^2}. \\
       \end{array}}
\]
\endgroup



%%====================================
\subparagraph{Программная реализация.}
%%====================================
Решение этой задачи может быть выражено на языке \lang{C++} следующим образом:\label{code:lsm}

\cppfile{01/src/least_squares.cpp}



%%================
\WhatToReadSection
%%================
\textcite{Stroustrup:2016:ru}: \textbf{главы~0, 1 и~2}



%%===============
\ExercisesSection
%%===============
\begin{exercise}
\item Настройте среду разработки программ на языке \lang{C++} (см. страницу \pageref{sect:workEnv}).


\item Создайте свой первый проект с традиционной программой \textenglish{Hello, World!}.

\smallskip
\emph{Совет}: воспользуйтесь графическими инструкциями из архива
\begin{flushleft}
  \yadisk{cpp-seminars/how-to-s/how-to\_create-project.zip}.
\end{flushleft}

Заголовочный файл \code{std\_lib\_facilities.h} из книги Страуструпа размещён в директории \yadisk{cpp-seminars/libraries}.


\item \textbf{NB!} Внимательно изучите и возьмите на вооружение <<горячие клавиши>> среды разработки. Начните осваивать печать вслепую.

\smallskip
\emph{Совет}: воспользуйтесь информацией из раздела на странице~\pageref{sect:typing}.
\end{exercise}
